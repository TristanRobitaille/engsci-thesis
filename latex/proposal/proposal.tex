\documentclass[12pt]{article}

% Packages
\usepackage[backend=biber, citestyle=ieee]{biblatex}
\usepackage[letterpaper, margin=0.75in]{geometry}
\usepackage{fancyhdr}
\usepackage{lipsum}

\addbibresource{references.bib}

% Frontpage matters
\lhead{ESC499 EngSci Thesis Proposal}
\title{Edge AI-optimized ASIC Prototype for Sleep Stage Detection}
\author{Tristan Robitaille (1006343397)}
\date{}

% Document
\begin{document}
\maketitle

\thispagestyle{fancy}

% Main text
\section{Background}
As reported by Chaput \textit{et al.} \cite{insomnia_prevalence}, insomnia impacts around 24\% of Canadians adult. Sleep stage detection (SSD), followed by neuromodulation, has been recently found by Yoon \cite{yoon2021neuromodulation} to be a promising
treatment against insomnia. However, state-of-art SSD techniques such as polysomnography are cumbersome to use (at least 19 sensors are required, as explained by Levin and Chauvel \cite{RUNDO2019381}), require clinical supervision and do not offer neuromodulation.
Thus, there is a need to develop a self-contained implanted brain-machine interface (BMI) device for SSD and neuromodulation. To maximize treatment potential, the device should be as small and portable as possible.

\section{Objectives}
To maintain a reasonable scope, this thesis project focusses on prototyping the compute software and hardware for sleep stage detection. Multiple authors \cite{dutt2023sleepxai, fu2021deep, eldele2021attention} have published high-accuracy results using a deep learning approach to SSD, and have done so with significantly fewer sensors 
than polysomnography. However, these AI models run on standard computers as software frameworks and are thus unsuitable for a lightweight integrated solution. Google sells small custom AI-accelerators (such as the Coral Edge TPU) that could run these AI models, but they still consume too much power (2W, \cite{coral_datasheet}) and do not readily integrate with
custom neuromodulation hardware.
Thus, the objective of this thesis is to demonstrate the superiority of a hardware implementation of an AI model compared to state-of-the-art SSD methods. The proposed solution should match or exceed the accuracy of traditional polysomnography and published models in the literrature while consuming less power than the currently available commercial hardware solutions.

\section{Methodology}
To achieve the aforementioned objectives, a three-step plan is proposed. First, a prototype Python model will developed, trained, evaluated and optimized until its accuracy surpasses existing solutions and size cannot be reduced further. Then, this model will be translated into C. Finally, the model and associated system architecture will be designed, implemented on a FPGA development board, benchmarked and continuously optimized for low power consumption. This high- to low-level approach will provide gradually increasing exposure to the computations involved and opportunities for design improvements while still providing relatively rapid debugging and optimization cycles. This final model will be benchmarked against published researched and an equivalent model running on Google Coral.

% Bibliography
\newpage
\printbibliography

% Questions for draft review
\newpage
\section{Questions for draft review}
\begin{enumerate}
    \item Is this proposal the place to discuss industry interest in this ASIC project? I would argue that it isn't because we should rather discuss the benefits to society rather than specific industry (they should be synonyms...), but I don't want to miss a good "selling point".
    \item Should I provide more specifics for the "improvements to AI model and hardware architecture" yet? I would not think so since this is only a proposal of the project and I do not want to overconstrain, but unsure if this is specific enough.
    \item I would like to mention a specific power budget, but is it too early to do so in a proposal? I think a number should be justified, so adding the number will consume some words. However, as a reader, I would still like an order of magnitude (do we want <1mW or <1W?).
    \item I'm wondering if I should include as an objective publishing in one of the journals you mentionned, but I suspect here that "objectives" refers more to the objectives of the solution rather than of course/project at large.
    \item The rubric mentions as an evaluation item "Specific method(s) identified, where possible". If this refers to the software tools and design resources used (Tensorflow, Compute Canada cluster, Xilinx Vivado and development board, SystemVerilog benchtests, Google Scholar, etc.), I can include them but I didn't as I felt that would overburden the text without providing significant insight on the merits of the projects to the reader.
    \item Is the development of the device justified enough (first paragraph)? I'd like to add some background on why it needs to be as small/low-power as possible. Would the eventual commercial device be implemented in the patient's skull or more likely used only at night with external electrodes (at home or clinic?)? Do we have figures on the target number of patients that could be treated with this?
    \item Any general comments on content, organization, prose, usage of references?
\end{enumerate}

\end{document}