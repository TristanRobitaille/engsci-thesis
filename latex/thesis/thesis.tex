\documentclass[12pt, hidelinks]{article}
\usepackage[a4paper, margin=1in]{geometry} % Set 1-inch border
\usepackage[backend=biber, citestyle=ieee]{biblatex}
\usepackage{setspace}
\usepackage{acronym}
\usepackage{float}
\usepackage{hyperref}
\usepackage{setspace}
\usepackage{lipsum} % Dummy text
\usepackage{multirow} % Table
\usepackage{tocloft}
\usepackage{quantikz}
\usepackage{booktabs}
\usepackage{fancyhdr} % Headers and footers

\addbibresource{../references.bib}
% Dots (leaders) in table of contents
\renewcommand{\cftsecleader}{\cftdotfill{\cftdotsep}}
\renewcommand{\cftsubsecleader}{\cftdotfill{\cftdotsep}}
\renewcommand{\cftsubsubsecleader}{\cftdotfill{\cftdotsep}}
\renewcommand{\thetable}{\Roman{table}}

\begin{document}
% Front cover
\title{Vision Transformer Accelerator ASIC for In-Ear Sleep Staging}
\author{Tristan Robitaille}
\newcommand{\supervisor}{Professor Xilin Liu}
\makeatletter
\renewcommand{\maketitle}{%
    \begin{titlepage}
        \onehalfspacing
        \begin{center}
            {\Large\textbf{\@title}\par}
            \vspace{2cm}
            by\par
            {\large{\@author}\par}
            \vspace{2cm}
            {\large Supervisor: \supervisor \\April 2024}
        \end{center}

        \vfill

        \begin{flushright}
        {\Huge\textbf{B.A.Sc. Thesis}}
        \end{flushright}

        \vspace{0.1\baselineskip}

        \begin{spacing}{0.4}
        \begin{flushright}
        \rule{3.25cm}{0.3pt}\\
        \rule{3.25cm}{0.3pt}\\
        \rule{3.25cm}{0.3pt}\\
        \rule{3.25cm}{0.3pt}
        \end{flushright}
        \vspace{-2\baselineskip}
        \rule{\textwidth}{0.3pt} 
        \rule{3.25cm}{0.3pt} \hspace{\textwidth-6.75cm} \rule{3.25cm}{0.3pt}\\
        \rule{3.25cm}{0.3pt} \hspace{\textwidth-6.75cm} \rule{3.25cm}{0.3pt}\\
        \rule{3.25cm}{0.3pt} \hspace{\textwidth-6.75cm} \rule{3.25cm}{0.3pt}\\
        \rule{3.25cm}{0.3pt} \hspace{\textwidth-6.75cm} \rule{3.25cm}{0.3pt}\\
        \rule{3.25cm}{0.3pt}\\
        \rule{3.25cm}{0.3pt}\\
        \rule{3.25cm}{0.3pt}\\
        \rule{3.25cm}{0.3pt}
        \vspace{2.5\baselineskip}
        \end{spacing}
        \begin{figure}[H]
            \includegraphics[width=10.5cm]{assets/div_engsci_logo.pdf}
        \end{figure}
    \end{titlepage}
}
\makeatother
\maketitle

% Flyleaf
\newpage
\thispagestyle{empty}
\vspace*{\fill}
\begin{center}
\Large
This page intentionally left blank.
\end{center}
\vspace*{\fill}

\onehalfspacing

% Title
\newpage
\begin{titlepage}
    \thispagestyle{empty}
    \centering
    {\LARGE\bfseries ESC499 Engineering Science Thesis\par}
    {\Large Vision Transformer Accelerator ASIC for In-Ear Sleep Staging\par}
    \vspace{8cm}
    {\Large Tristan Robitaille\par}
    {\textit{Student number}: 1006343397\par}
    {\textit{Email}: tristan.robitaille@mail.utoronto.ca\par}
    \vspace{5cm}
    {\large Supervisor: Professor Xilin Liu\par}
    {\textit{Email}: xilinliu@ece.utoronto.ca\par}
    \vspace{2cm}
    {\large April 12th, 2024\par}
    \vspace{2cm}
    \begin{center}
        \copyright\ 2024 Tristan Robitaille
    \end{center}        
\end{titlepage}
\newpage

\pagenumbering{roman}
% Abstract
\section*{Abstract}
    \lipsum[1]
\newline
\newline
{\bf Keywords:} Sleep staging, ASIC accelerator, vision transformer, computer architecture
\newpage

% Acknowledgements
\section*{Acknowledgements}
I would like to express my gratitude to my supervisor, Prof. Xilin Liu, for his guidance and support throughout the project. He has given me the freedom to explore new ideas and had provided me with the
support and tools I needed.

I would also like to thank my father, Claude Robitaille, for letting me remotely use his workstation to train the model and run the accuracy study. He has also helped review the code for the functional simulation.

In addition, I owe much to the professors who have taught me the fundamentals of computer architecture at the University of Toronto - Profs. Jason Anderson, Natalie Enright-Jerger, Andreas Moshovos and Mark C. Jeffrey.

Throughout this project, I have made extensive use the Compute Canada cluster, which has provided me with the computational resources I needed to run the simulations and train the model. I would like to thank the 
staff at Compute Canada for their initiative. I am also appreciative of the tools provided by the Canadian Microelectronics Corporation, which have been instrumental in the hardware implementation of the accelerator.

I would also like to acknowledge the work of Professors Lisa Romkey and Alan Chong who organized this thesis project for us, ensuring a structured and productive environment.

Finally, I would like to thank my family and friends for their support and encouragement throughout this project. I am grateful for their patience and understanding during this time.

\newpage

% Table of Contents
\tableofcontents
\newpage

% List of Figures
\listoffigures
\newpage

% List of Tables
\listoftables
\newpage

% List of abbreviations
\section*{List of Abbreviations}
\begin{acronym}
    {\small\setstretch{0.8}
    \acro{adc}[ADC]{Analog-to-Digital Converter}
    \acro{afe}[AFE]{Analog Front-End}
    \acro{asic}[ASIC]{Application-Specific Integrated Circuit}
    \acro{cim}[CiM]{Compute-in-Memory}
    \acro{cmos}[CMOS]{Complimentary Metal Oxide Semiconductor}
    \acro{csv}[CSV]{Comma-Separated Values}
    \acro{eeg}[EEG]{Electroencephalography}
    \acro{hdl}[HDL]{Hardware Description Language}
    \acro{hdf5}[HDF5]{Hierarchical Data Format 5}
    \acro{ip}[IP]{Intellectual Property}
    \acro{pe}[PE]{Processing Element}
    \acro{ppa}[PPA]{Power, Performance, and Area}
    \acro{mac}[MAC]{Mulitiply-Accumulate}
    \acro{mass}[MASS]{Montreal Archive of Sleep Studies}
    \acro{mhsa}[MHSA]{Multi-Head Self-Attention}
    \acro{mlp}[MLP]{Multi-Layer Perceptron}
    \acro{psg}[PSG]{Polysomnography}
    \acro{rtl}[RTL]{Register Transfer Level}
    \acro{tsmc}[TSMC]{Taiwan Semiconductor Manufacturing Company}
    \acro{vcd}[VCD]{Value Change Dump}
    \acro{fsm}[FSM]{Finite State Machine}
    }
\end{acronym}
\newpage

% Body
\pagenumbering{arabic}
    See~\cite{liu2021edge}.
    I am making an \ac{asic}. It's small, low-power and fast. It's better than Google's.
\section{Introduction}
\lipsum[1]

\newpage
\section{Background}
\lipsum[1]

\newpage
\section{How to Design an AI Accelerator}
\lipsum[1]
\subsection{Model Prototyping}
\lipsum[1]
\subsection{Accelerator Functional Simulation}
\lipsum[1]
\subsection{Accelerator Hardware Implementation}
\lipsum[1]

\newpage
\section{Design Overview}
\lipsum[1]

\subsection{Vision Transformer}
\lipsum[1]
\subsection{Accelerator Architecture}
\subsubsection{Centralized vs. Distributed Architecture}
\subsubsection{Master Architecture}
\subsubsection{Data and Control Bus}
\subsubsection{Compute-in-Memory: Fixed-Point Accuracy}
\subsubsection{Compute-in-Memory: Memory}
\subsubsection{Compute-in-Memory: Compute Modules}

\subsection{A Note About Software-Hardware Co-Design}

\lipsum[1]

\newpage
\section{Evaluation of Performance Metrics}
\lipsum[1]
\subsection{Vision Transformer}
\lipsum[1]
\subsection{Accelerator}
\lipsum[1]

\newpage
\section{Future Work}
\lipsum[1]

\newpage
\section{Conclusion}
\lipsum[1]

\newpage

% Bibliography
\printbibliography
\newpage

% Appendices
\appendix
\section{Bus Operations}
\label{app:bus_ops}
This section details the instructions that can be performed on the bus in more details than is warranted in the main body of the thesis. Table \ref{tab:bus_ops} describes
each instruction along with the fields on the bus.

\begin{sidewaystable}
    \centering
    \renewcommand{\arraystretch}{1.2} % Vertical spacing
    \setlength{\arrayrulewidth}{1.5pt} % Thickness of vertical lines
    \caption{Bus operations and their fields}
    \begin{tabular}{@{} p{6.5cm}lllllll @{}}
        \toprule
        Opcode                                      & Description                       & Sender        & \texttt{ID}   & \texttt{Data[0]}  & \texttt{Data[1]}  & \texttt{Data[2]} \\\midrule
        \texttt{NOP}                                & No instruction                    & All           & -             & -                 & -                 & - \\
        \texttt{PATCH\_LOAD\_BROADCAST\_START\_OP}  & Start loading an EEG patch        & Master        & 0-63          & \texttt{tx\_addr} & Length            & \texttt{rx\_addr} \\
        \texttt{PATCH\_LOAD\_BROADCAST\_OP}         & \ac{eeg} patch data               & \ac{cim}      & 0-63          & Data              & Data              & Data \\
        \texttt{DENSE\_BROADCAST\_START\_OP}        & Start dense broadcast             & Master        & 0-63          & \texttt{tx\_addr} & Length            & \texttt{rx\_addr} \\
        \texttt{DENSE\_BROADCAST\_DATA\_OP}         & Dense broadcast data              & \ac{cim}      & 0-63          & Data              & Data              & Data \\
        \texttt{PARAM\_STREAM\_START\_OP}           & Start streaming weights           & Master        & 0-63          & \texttt{tx\_addr} & Length            & - \\
        \texttt{PARAM\_STREAM\_OP}                  & Weight data                       & Master        & 0-63          & Data              & Data              & Data \\
        \texttt{TRANS\_BROADCAST\_START\_OP}        & Start transpose broadcast         & Master        & 0-63          & \texttt{tx\_addr} & Length            & - \\
        \texttt{TRANS\_BROADCAST\_DATA\_OP}         & Transpose data broadcast          & \ac{cim}      & 0-63          & Data              & Data              & Data \\
        \texttt{PISTOL\_START\_OP}                  & \ac{cim} to execute next step     & Master        & -             & -                 & -                 & - \\
        \texttt{INFERENCE\_RESULT\_OP}              & Contains inferred sleep stage     & \ac{cim} \#0  & 0             & Sleep stage       & -                 & - \\
    \end{tabular}
    \label{tab:bus_ops}
\end{sidewaystable}

\section{Codebase Statistics}
It may be interesting to the reader to appreciate the size of the codebase needed to develop a project of similar scale. The code for this project is available 
in my \href{https://github.com/TristanRobitaille/engsci-thesis}{GitHub repository}. The following table provides a breakdown of the number of lines of code in the project.

\begin{table}[ht]
    \centering
    \renewcommand{\arraystretch}{1.2} % Vertical spacing
    \setlength{\arrayrulewidth}{1.5pt} % Thickness of vertical lines
    \caption{Line and file count per file type in the codebase}
    \begin{tabular}{@{} p{4cm}cccr @{}}
        \toprule
        File type       & File count    & Line count    & Percent of total & \\\midrule
        Python          & 12            & 3000          & 33.7\% \\
        SystemVerilog   & 12            & 2500          & 30.4\% \\
        C++             & 12            & 1250          & 18.9\% \\
        TeX             & 12            & 670           & 8.2\%  \\
        Shell           & 12            & 300           & 4.3\%  \\
        Other           & 12            & 20            & 4.5\%  \\\midrule
        Total           & 60            & 13,000        & 100\%  \\
        \hline
    \end{tabular}
    \label{tab:line_cnt}
\end{table}

In addition, there have been 200 commits to the repository.

\newpage
\section{Reflection on Learnings and Experience Gained}

% Flyleaf
\newpage
\thispagestyle{empty}
\vspace*{\fill}
\begin{center}
\Large
This page intentionally left blank.
\end{center}
\vspace*{\fill}

% Back cover
\newpage

\end{document}
