\appendix
\section{Bus Operations}
\label{app:bus_ops}
This section details the instructions that can be performed on the bus in more details than is warranted in the main body of the thesis. Table \ref{tab:bus_ops} describes
each instruction along with the fields on the bus.

\begin{sidewaystable}
    \centering
    \renewcommand{\arraystretch}{1.2} % Vertical spacing
    \setlength{\arrayrulewidth}{1.5pt} % Thickness of vertical lines
    \caption{Bus operations and their fields}
    \begin{tabular}{@{} p{6.5cm}lllllll @{}}
        \toprule
        Opcode                                      & Description                       & Sender        & \texttt{ID}   & \texttt{Data[0]}  & \texttt{Data[1]}  & \texttt{Data[2]} \\\midrule
        \texttt{NOP}                                & No instruction                    & All           & -             & -                 & -                 & - \\
        \texttt{PATCH\_LOAD\_BROADCAST\_START\_OP}  & Start loading an EEG patch        & Master        & 0-63          & \texttt{tx\_addr} & Length            & \texttt{rx\_addr} \\
        \texttt{PATCH\_LOAD\_BROADCAST\_OP}         & \ac{eeg} patch data               & \ac{cim}      & 0-63          & Data              & Data              & Data \\
        \texttt{DENSE\_BROADCAST\_START\_OP}        & Start dense broadcast             & Master        & 0-63          & \texttt{tx\_addr} & Length            & \texttt{rx\_addr} \\
        \texttt{DENSE\_BROADCAST\_DATA\_OP}         & Dense broadcast data              & \ac{cim}      & 0-63          & Data              & Data              & Data \\
        \texttt{PARAM\_STREAM\_START\_OP}           & Start streaming weights           & Master        & 0-63          & \texttt{tx\_addr} & Length            & - \\
        \texttt{PARAM\_STREAM\_OP}                  & Weight data                       & Master        & 0-63          & Data              & Data              & Data \\
        \texttt{TRANS\_BROADCAST\_START\_OP}        & Start transpose broadcast         & Master        & 0-63          & \texttt{tx\_addr} & Length            & - \\
        \texttt{TRANS\_BROADCAST\_DATA\_OP}         & Transpose data broadcast          & \ac{cim}      & 0-63          & Data              & Data              & Data \\
        \texttt{PISTOL\_START\_OP}                  & \ac{cim} to execute next step     & Master        & -             & -                 & -                 & - \\
        \texttt{INFERENCE\_RESULT\_OP}              & Contains inferred sleep stage     & \ac{cim} \#0  & 0             & Sleep stage       & -                 & - \\
    \end{tabular}
    \label{tab:bus_ops}
\end{sidewaystable}

\section{Codebase Statistics}
It may be interesting to the reader to appreciate the size of the codebase needed to develop a project of similar scale. The code for this project is available 
in my \href{https://github.com/TristanRobitaille/engsci-thesis}{GitHub repository}. The following table provides a breakdown of the number of lines of code in the project.

\begin{table}[ht]
    \centering
    \renewcommand{\arraystretch}{1.2} % Vertical spacing
    \setlength{\arrayrulewidth}{1.5pt} % Thickness of vertical lines
    \caption{Line and file count per file type in the codebase}
    \begin{tabular}{@{} p{4cm}cccr @{}}
        \toprule
        File type       & File count    & Line count    & Percent of total & \\\midrule
        Python          & 12            & 3000          & 33.7\% \\
        SystemVerilog   & 12            & 2500          & 30.4\% \\
        C++             & 12            & 1250          & 18.9\% \\
        TeX             & 12            & 670           & 8.2\%  \\
        Shell           & 12            & 300           & 4.3\%  \\
        Other           & 12            & 20            & 4.5\%  \\\midrule
        Total           & 60            & 13,000        & 100\%  \\
        \hline
    \end{tabular}
    \label{tab:line_cnt}
\end{table}

In addition, there have been 200 commits to the repository.

\newpage
\section{Reflection on Learnings and Experience Gained}