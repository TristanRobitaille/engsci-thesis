\appendix
\section{Bus Operations}
\label{app:bus_ops}
This section details the instructions that can be performed on the bus in more details than is warranted in the main body of the thesis. Table \ref{tab:bus_ops} describes
each instruction along with the fields on the bus.

\begin{sidewaystable}
    \centering
    \renewcommand{\arraystretch}{1.2} % Vertical spacing
    \setlength{\arrayrulewidth}{1.5pt} % Thickness of vertical lines
    \caption{Bus operations and their fields}
    \begin{tabular}{@{} p{6.5cm}lllllll @{}}
        \toprule
        Opcode                                      & Description                       & Sender        & \texttt{ID}   & \texttt{Data[0]}  & \texttt{Data[1]}  & \texttt{Data[2]} \\\midrule
        \texttt{NOP}                                & No instruction                    & All           & -             & -                 & -                 & - \\
        \texttt{PATCH\_LOAD\_BROADCAST\_START\_OP}  & Start loading an EEG patch        & Master        & 0-63          & \texttt{tx\_addr} & Length            & \texttt{rx\_addr} \\
        \texttt{PATCH\_LOAD\_BROADCAST\_OP}         & \ac{eeg} patch data               & \ac{cim}      & 0-63          & Data              & Data              & Data \\
        \texttt{DENSE\_BROADCAST\_START\_OP}        & Start dense broadcast             & Master        & 0-63          & \texttt{tx\_addr} & Length            & \texttt{rx\_addr} \\
        \texttt{DENSE\_BROADCAST\_DATA\_OP}         & Dense broadcast data              & \ac{cim}      & 0-63          & Data              & Data              & Data \\
        \texttt{PARAM\_STREAM\_START\_OP}           & Start streaming weights           & Master        & 0-63          & \texttt{tx\_addr} & Length            & - \\
        \texttt{PARAM\_STREAM\_OP}                  & Weight data                       & Master        & 0-63          & Data              & Data              & Data \\
        \texttt{TRANS\_BROADCAST\_START\_OP}        & Start transpose broadcast         & Master        & 0-63          & \texttt{tx\_addr} & Length            & - \\
        \texttt{TRANS\_BROADCAST\_DATA\_OP}         & Transpose data broadcast          & \ac{cim}      & 0-63          & Data              & Data              & Data \\
        \texttt{PISTOL\_START\_OP}                  & \ac{cim} to execute next step     & Master        & -             & -                 & -                 & - \\
        \texttt{INFERENCE\_RESULT\_OP}              & Contains inferred sleep stage     & \ac{cim} \#0  & 0             & Sleep stage       & -                 & - \\
    \end{tabular}
    \label{tab:bus_ops}
\end{sidewaystable}

\section{Codebase Statistics}
It may be interesting to the reader to appreciate the size of the codebase needed to develop a project of similar scale. The code for this project is available 
in my \href{https://github.com/TristanRobitaille/engsci-thesis}{GitHub repository}. The following table provides a breakdown of the number of lines of code in the project.

\begin{table}[ht]
    \centering
    \renewcommand{\arraystretch}{1.2} % Vertical spacing
    \setlength{\arrayrulewidth}{1.5pt} % Thickness of vertical lines
    \caption{Line and file count per file type in the codebase}
    \begin{tabular}{@{} p{4cm}cccr @{}}
        \toprule
        File type       & File count    & Line count    & Percent of total & \\\midrule
        Python          & 12            & 3000          & 33.7\% \\
        SystemVerilog   & 12            & 2500          & 30.4\% \\
        C++             & 12            & 1250          & 18.9\% \\
        TeX             & 12            & 670           & 8.2\%  \\
        Shell           & 12            & 300           & 4.3\%  \\
        Other           & 12            & 20            & 4.5\%  \\\midrule
        Total           & 60            & 13,000        & 100\%  \\
        \hline
    \end{tabular}
    \label{tab:line_cnt}
\end{table}

In addition, there have been 200 commits to the repository.

\newpage
\section{Reflection on Learnings and Experience Gained}
This project has been a significant learning experience for me. I have learned a great deal about artifical intelligence and the design and implementation of hardware systems.
I'm including this section to formalize reflections on experience gained, experience I wasn't able to gain and what I would have done differently.

\subsection{Acquired Experience}
Firstly, I gained good knownledge in the basics of artificial intelligence through Andrew Ng's Deep Learning Specialization on Coursera and practical experience using TensorFlow.
I also further reinforced my skills in C++ and SystemVerilog and was able to use industry-standard tools such as Synopsys Design Compiler and ARM Artisan IP. I developed my own
local workflow for developing RTL with Verilator, CocoTB and GTKWave, whose flexibility will be a great asset and enabler in my future projects.

That being said, the largest takeway from this project is a direct appreciation for the complexity of hardware-software co-design. This project took me from the high-level frameworks
of Python to bit-level arithmetic, \ac{fsm} and cycle-level parallelism. Through this, I've realized that owning the full-stack is powerful and gives significant design freedom to optimize
the system. In turn, this can prove destabilizing as essentially all aspects of the design have compounding pros and cons. It is thus critical to develop flexible, accurate and actionable
functional simulations to evaluate small different aspects of th design before committing to a full implementation. I am glad to have done that to some extent with the C++ model 
and various Python studies, but, in retrospect, more time should have been spent desiging and obtaining proxy metrics to determine the ideal high-level architecture. For instance, I 
did not need to design the full system before being able to obtain \ac{ppa} figures for the memory and compute units. This would have allowed me to make more informed decisions about
the architecture and potentially save time in the long run. However, I think such learning can only be appreciated once an architect goes through the full design cycle at least once,
so I am glad to have had the opportunity to earn this wisdom, which I will carry in future projects. I'm realizing that a informed, complete, analysis of a design without necessarily
implementing it is more insightful and and impactful than a somewhat underdiscussed but working design

\subsection{Topics Warranting Further Exploration}
I would have liked to have had more time to explore design synthesis and implementation. I only managed to make use of Design Compiler and Artisan IP for about six weeks, and I would have
liked to be able to learn more about its different features, optimization strategies and how to contol it. I would have also liked to have had more time to explore the impact of different
of memory compilation. Finally, this project stopped at design synthesis. Implementation and post-routing simulations are another area that I did not have time to explore.

\subsection{Alternative Approaches for Consideration}
In my opinion, the three steps plan presented in Section \ref{sec:methods} was a good approach to the project. However, it was too coarse and I would add two steps: a step between the model
and the functional simulation to translate the model to software-style C++ instead of hardware-style C++. This would have allowed to more easily test things such as fixed-point strategies
and collect metrics such as exact number of different types of operations and memory accesses without getting slowed down by hardware-style coding. I would also write the functional simulation
from the bottom up along with coding hardware modules in parallel. This would allow progressively more precise evaluation of the \ac{ppa} metrics and provide more time to make architecture 
decisions.