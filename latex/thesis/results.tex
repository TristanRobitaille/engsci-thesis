\subsection{Model and Hardware Results}
This Section presents and discusses the most salient aggregate high-level results of the model and hardware design as shown in Table \ref{tab:high_level_results}.
It can be seen that the model meets the requirements summarized in Table \ref{tab:design_goals}. Indeed, the model size (63.18\si{\kilo\byte}) is below the
125\si{\kilo\byte} constraint, and the accuracy (82.9\%) is higher than the 80\% target. It is also nearly $32\times$ smaller than the smallest state-of-the-art
model for automatic sleep staging presented by Eldele \textit{et al.}, which has 500,000 32-bit floating point parameters \cite{eldele2021attention}.

The ASIC accelerator meets some of the requirements. The inference latency (6.97\si{\milli\second}) is well below the maximum allowed latency of 3\si{\second}.
The \ac{pe} utilization (50.8\%) of the accelerator suggests that there is non-negligible time overhead (mostly from inter-\ac{cim} communication) that could be
reduced in future work. The \ac{pe} utilization computes the amount of time that any one compute module is busy or refreshing its output relative to the total inference
time. The toral area is 3.86\si{\square\milli\meter}, which is above the target of 2\si{\square\milli\meter}. The effective power of 3.046 \si{\milli\watt} is below
the ceiling of 5\si{\milli\watt}.

The effective total power considers the effects of power gating, which is a common technique used in modern processors to reduce power consumption. The idea
is to turn off power to the ASIC accelerator when it is not in use through one or more low-side \ac{nmos} transistor (known as a ``sleep transistor''). Investigating
this technique for 65nm \ac{cmos} technology, Sathanur \textit{et al.} found that leakage power can be reduced by up to 95\% using power gating \cite{sathanur2008quantifying}.
The effective total power is calculated as the sum of the average dynamic power from inference energy and the leakage power pro-rated with power gating. It assumes a
sleep epoch duration of 30\si{\second}. The average dynamic power is computed for an inference with the techniques of analyzing a \ac{vcd} file as described in 
Section \ref{sec:arch}.

With an inference latency of 6.97\si{\milli\second} and a sleep epoch duration of 30\si{\second}, the ASIC accelerator only runs for 0.0232\% of the time, meaning
that the effective leakage power is reduced by up to 94.98\%, reducing the effective leakage power to 2.947\si{\milli\watt}. The accelerator consumes a total of
63.07\si{\milli\watt} during inference, which brings the effective power consumption to 3.046\si{\milli\watt} with power gating. The energy per inference is
433.29\si{\micro\joule}. The $f_{max}$ is 758\si{\mega\hertz}, which is above the target of 200\si{\mega\hertz}. This gives us ample leeway to modify the design to reduce
power consumption or area further. The $f_{max}$ is limited by the multiplier module as shown in Table \ref{tab:compute_modules}.

\begin{table}[ht]
    \centering
    \renewcommand{\arraystretch}{1.2} % Vertical spacing
    \setlength{\arrayrulewidth}{1.5pt} % Thickness of vertical lines
    \caption{Principal results of the model and hardware design}
    \begin{tabularx}{0.8\textwidth}{Xlc}
        \toprule
        Metric                      & Value                         & Meets requirement? \\\midrule
        31-fold accuracy            & 82.9\%                        & Yes   \\
        \# of parameters            & 31,589                        & Yes   \\
        Size                        & 63.18\si{\kilo\byte}          & Yes   \\ \bottomrule 
        %%%%%%%%%%%%%%%%%%%%%%
        Inference latency           & 6.87\si{\milli\second}        & Yes   \\
        Area                        & 3.86\si{\square\milli\meter}  & No    \\
        \ac{pe} utilization         & 50.8\%                        & N/A   \\
        Leakage power               & 56.48\si{\milli\watt}         & N/A   \\
        Average dynamic power       & 6.59\si{\milli\watt}          & N/A   \\
        Effective total power       & 3.046\si{\milli\watt}         & Yes   \\
        Energy/inference            & 433.29\si{\micro\joule}       & N/A   \\
        $f_{Max}$                   & 758\si{\mega\hertz}           & Yes   \\ \bottomrule
    \end{tabularx}
    \label{tab:high_level_results}
\end{table}

\subsection{Comparison Against the Coral Edge TPU}
The Coral Edge \ac{tpu} is a low-power USB accelerator designed by Google optimized for inference. It consumes up to 2\si{\watt} and can process up to 4TOPS \cite{coral_datasheet}.
It represents one of the AI accelerator with the lower power consumption available in the market, and as such is a good reference point for comparison. The TensorFlow
model was converted to a TensorFlow Lite model and run 2000 times on the Coral Edge \ac{tpu} to compare the results. The mean inference time was 0.754\si{\milli\second}
with a standard deviation of 0.038\si{\milli\second}. Although the Coral Edge \ac{tpu} is faster than the ASIC accelerator, it consumes too much power to be used for this
project. Furthermore, the integrated circuit measures roughly 5\si{\milli\meter} by 5\si{\milli\meter}, which is too large for the target application. It also doesn't not readily
integrate with the rest of the system, such as the \ac{afe} for sensing, the external memory for weight storage or the neuromodulation coil driver.